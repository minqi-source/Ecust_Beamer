\documentclass{beamer}

% ========== 基础包 ==========
\usepackage{ctex}
\usepackage{hyperref}
\usepackage{graphicx}
\usepackage{tikz}
\usepackage{eso-pic}  % 用于在每页添加logo
% \usepackage[T1]{fontenc} % T1 encoding is less relevant when using XeLaTeX/LuaLaTeX with Unicode fonts, can often be removed or might be handled by ctex

% ========== 参考文献支持 ==========
% 注意:您原文件中使用了传统的 BibTeX 方法 (\bibliographystyle, \bibliography)
% 如果您希望使用更现代、功能更强大的 biblatex (如我们在第一个回复中建议的),
% 需要移除这两行,并添加 biblatex 的设置 (\usepackage[...]{biblatex}, \addbibresource{...})
% 并使用 biber 进行编译。
\bibliographystyle{plain}  % 使用 plain 样式,您也可以选择其他样式如 unsrt, alpha 等

% ========== 数学相关包 ==========
\usepackage{amsmath}  % 基础数学支持 (保留,提供环境)
\usepackage{amssymb}  % 数学符号 (保留,提供一些基本符号和环境)
\usepackage{amsthm}   % 定理环境 (保留)
% \usepackage{latexsym} % 额外数学符号 (移除,符号通常已包含)
% \usepackage{mathptmx}  % 基础数学字体支持 (移除,与 unicode-math 冲突)
% \usepackage{mathpazo} % 数学字体支持 (移除,与 unicode-math 冲突)
\usepackage{unicode-math} % 使用 unicode 数学字体
\setmathfont{Latin Modern Math}  % 使用 XITS Math 字体,它对数学符号支持更好

% ========== 算法相关包 ==========
\usepackage{algorithm}
\usepackage{algpseudocode}

% ========== 图形相关包 ==========
\usepackage{pstricks}
\usepackage{caption}
\usepackage{subcaption}

% ========== 其他工具包 ==========
\usepackage{multicol}
\usepackage{booktabs}
\usepackage{calligra}
\usepackage{listings}
\usepackage{stackengine}

% ========== 自定义命令 ==========
\def\cmd#1{\texttt{\color{red}\footnotesize $\backslash$#1}}
\def\env#1{\texttt{\color{blue}\footnotesize #1}}

% ========== 颜色定义 ==========
\definecolor{deepblue}{rgb}{0,0,0.5}
\definecolor{deepred}{rgb}{0.6,0,0}
\definecolor{deepgreen}{rgb}{0,0.5,0}
\definecolor{halfgray}{gray}{0.55}
\definecolor{lightorange}{rgb}{1,0.9,0.635} % 新增颜色定义,RGB值为(255, 230, 162),转换为0-1范围的值
\definecolor{purple1}{RGB}{146, 232, 232}

% ========== 列表设置 ==========
\lstset{
    basicstyle=\ttfamily\small,
    keywordstyle=\bfseries\color{deepblue},
    emphstyle=\ttfamily\color{deepred},
    stringstyle=\color{deepgreen},
    numbers=left,
    numberstyle=\small\color{halfgray},
    rulesepcolor=\color{red!20!green!20!blue!20},
    frame=shadowbox,
}

% ========== 文档信息 (请替换为您的信息) ==========
\author{Your Name}
\title{Your Presentation Title 标题}
\subtitle{Optional Subtitle 副标题}
\institute{Your University/Institution 单位/学院}
\date{\today} % 或具体日期

% ========== 主题设置 (保留原主题设置) ==========
\usepackage{ECUST} % 保留原主题包

% ========== 颜色设置 (保留原颜色设置) ==========
\definecolor{titlecolor}{RGB}{44, 103, 167}  % 定义标题栏颜色
\setbeamercolor{frametitle}{bg=titlecolor,fg=white}  % 设置标题栏背景色和文字颜色

% ========== Logo设置 (保留原Logo设置结构, 请替换为您的logo文件路径) ==========
% 普通页面的logo设置
\setbeamertemplate{frametitle}{
    \nointerlineskip
    \begin{beamercolorbox}[wd=\paperwidth,ht=2.2ex,dp=0.7ex,leftskip=0.3cm,rightskip=0cm]{frametitle}
        \hbox to \paperwidth{
            \strut
            \insertframetitle
            \hfill
            % 替换为您的logo文件路径
            \raisebox{-0.132cm}{
                \includegraphics[height=2.9ex]{images/ecust_logo7.pdf} % 示例logo文件
            }
            \hspace{0.0001cm}  % 添加一点点右边距
        }
    \end{beamercolorbox}
}

% 首页设置
\setbeamertemplate{title page}{
    \begin{center}
        \vspace*{1em}
        {\usebeamerfont{title}\inserttitle\par}
        {\usebeamerfont{subtitle}\insertsubtitle\par}
        \vspace{1.5em}
        \includegraphics[width=0.2\textwidth]{images/ecust_logo1.pdf}\par % 示例logo文件
        \vspace{0.5em}
        {\usebeamerfont{author}\insertauthor\par}
        {\usebeamerfont{institute}\insertinstitute\par}
        {\usebeamerfont{date}\insertdate\par}
        \vspace{-0.5em}
    \end{center}
}

% 目录页的logo设置
\setbeamertemplate{section page}{
    \begin{beamercolorbox}[wd=\paperwidth,ht=2.2ex,dp=0.7ex,leftskip=0.3cm,rightskip=0cm]{frametitle}
        \hbox to \paperwidth{
            \strut
            \hfill
            \raisebox{-0.132cm}{
                \includegraphics[height=2.9ex]{images/logo.pdf} % 示例logo文件
            }
            \hspace{0.05cm}  % 添加一点点右边距
        }
    \end{beamercolorbox}
    \vfill
    \begin{center}
        {\usebeamerfont{section name}\insertsection\par}
    \end{center}
    \vfill
}

% ========== 文档开始 ==========
\begin{document}

% ========== 标题页 ==========
\begin{frame}
    \titlepage
\end{frame}

% ========== 目录页 ==========
\begin{frame}{Outline} % 目录页标题
    \tableofcontents[sectionstyle=show,
    subsectionstyle=show/shaded/hide,
    subsubsectionstyle=show/shaded/hide
    ]
\end{frame}

% ========== 示例 Section ==========
\section{Section Title 1} % 示例章节标题
\begin{frame}{Frame Title 1.1} % 示例 Frame 标题
    \begin{block}{Block Title:}
		\begin{itemize}
        \item Placeholder concept 1111
        \item Placeholder concept 2
        \item Placeholder concept 3
        \item Placeholder concept 4
        \item Placeholder concept 5
        \item Placeholder concept 6
    \end{itemize}
	\end{block}
\end{frame}

\begin{frame}{Frame Title 1.2} % 示例 Frame 标题
\begin{alertblock}{Alert Block Title:} Placeholder challenge summary.\end{alertblock}
    \begin{columns}[T]
        \column{0.55\textwidth}
        \begin{itemize}
        	\item Placeholder point A
            \item Placeholder point B
            \item Placeholder point C
        \end{itemize}

        \column{0.45\textwidth}
        \begin{figure}
            \centering
            % Replace with your image path
            \includegraphics[width=0.95\textwidth]{images/logo.pdf} % 示例图片文件
            \caption{Placeholder Figure Caption 1}
        \end{figure}
    \end{columns}
    \vspace{-1em}
    \begin{exampleblock}{\small Example Block Title}\scriptsize Placeholder example text.\end{exampleblock}
\end{frame}

\begin{frame}{Frame Title 1.3} % 示例 Frame 标题
\begin{alertblock}{Alert Block Title:}
    \vspace{1em}
    \begin{columns}
		\column{0.5\textwidth}
			\textbf{Placeholder Topic A:}
            \begin{itemize}
                \item Placeholder point 1
                \item Placeholder point 2
                \item Placeholder point 3
            \end{itemize}
		\column{0.5\textwidth}
			\textbf{Placeholder Topic B:}
            \begin{itemize}
                \item Placeholder point 1
                \item Placeholder point 2
                \item Placeholder point 3
            \end{itemize}
	\end{columns}
\end{alertblock}
\end{frame}

\begin{frame}{Frame Title 1.4} % 示例 Frame 标题
Placeholder text before list: \textcolor{blue}{Placeholder text}, main contributions are:
    \begin{itemize}
    	\item Placeholder contribution 1
        \item Placeholder contribution 2
        \item Placeholder contribution 3
        \item Placeholder contribution 4
        \item Placeholder contribution 5
    \end{itemize}
\end{frame}

% ========== 示例 Section: Methodology ==========
\section{Section Title 2} % 示例章节标题
\subsection{Subsection Title 2.1} % 示例子章节标题

\begin{frame}{Frame Title 2.1} % 示例 Frame 标题
    \begin{columns}[T]
        \column{0.5\textwidth}
        \begin{block}{Block Title}
            \begin{displaymath}
                \begin{aligned}
                    \min_\theta F(\theta) &= \sum_{s \in S} \frac{n_s}{n} F_s(\theta) \\
                    &= \sum_{s \in S} \frac{n_s}{n} \frac{1}{n_s} \sum_{i \in D_s} \ell(\theta; x_i)
                \end{aligned}
            \end{displaymath}
            \vspace{-0.8em}
        \end{block}
        Placeholder explanation text.

        \column{0.5\textwidth}
        \begin{table}
            \centering \tiny
            \caption{\tiny Placeholder Table Caption}
            \begin{tabular}{cl}
                \toprule
                \textbf{符号} & \textbf{含义} \\
                \midrule
                $S$ & Placeholder meaning 1 \\
                $U$ & Placeholder meaning 2 \\
                $n_{s,u}$ & Placeholder meaning 3 \\
                $C$ & Placeholder meaning 4 \\
                $\sigma$ & Placeholder meaning 5 \\
                $\mathcal{N}(0,\sigma^2)$ & Placeholder meaning 6 \\
                \bottomrule
            \end{tabular}
        \end{table}
        \begin{alertblock}{Alert Block Title}
            Placeholder challenge summary.
        \end{alertblock}
    \end{columns}
\end{frame}

\begin{frame}{Frame Title 2.2} % 示例 Frame 标题
    \begin{definition}[Definition Title 1]
        Placeholder definition text 1.
    \end{definition}

    \begin{itemize}\scriptsize
        \item \textbf{Placeholder:} Item 1
        \item \textbf{Placeholder:} Item 2
        \item \textbf{Placeholder:} Item 3
    \end{itemize}

    \begin{definition}[Definition Title 2]
        Placeholder definition text 2.
    \end{definition}

    \begin{itemize}\scriptsize
        \item Placeholder item A
        \item Placeholder item B
        \item Placeholder item C
    \end{itemize}
\end{frame}

\begin{frame}{Frame Title 2.3} % 示例 Frame 标题
    \begin{definition}[Definition Title]
        Placeholder definition text.
    \end{definition}

    \begin{columns}[T]
        \column{0.55\textwidth}
        \begin{itemize}
            \item \textbf{Placeholder:} Item 1
            \item \textbf{Placeholder:} Item 2
            \item \textbf{Placeholder:} Item 3
        \end{itemize}

        \column{0.45\textwidth}
        \begin{table}
            \caption{\tiny Placeholder Comparison Table Caption}
            \centering \tiny
            \begin{tabular}{lcc}
                \toprule
                \textbf{特性} & \textbf{记录级} & \textbf{用户级} \\
                \midrule
                保护对象 & Placeholder 1A & Placeholder 1B \\
                敏感度 & Placeholder 2A & Placeholder 2B \\
                噪声量 & Placeholder 3A & Placeholder 3B \\
                实现难度 & Placeholder 4A & Placeholder 4B \\
                隐私保障 & Placeholder 5A & Placeholder 5B \\
                跨孤岛适用性 & Placeholder 6A & Placeholder 6B \\
                \bottomrule
            \end{tabular}
        \end{table}
    \end{columns}
\end{frame}

\begin{frame}{Frame Title 2.4 (Algorithm 1)} % 示例 Frame 标题
    \begin{algorithm}[H]
        \caption{Placeholder Algorithm 1}
        \begin{algorithmic}[1]\footnotesize
            \For{each silo $s \in S$}
                \State Perform local training
                \State Compute model update $\Delta_s^t$
                \State \colorbox{lightorange}{Apply clipping to $\Delta_s^t$}
                \State \colorbox{lightorange}{Add Gaussian noise $\mathcal{N}(0, \sigma^2 C^2|S|)$} \Comment{Placeholder comment}
            \EndFor
            \State Aggregate noisy updates
        \end{algorithmic}
    \end{algorithm}
    \vspace{-1em}
    \begin{alertblock}{Limitations:}
    \begin{itemize}
            \item Limitation A
            \item Limitation B
            \item Limitation C
    \end{itemize}
    Summary of limitations.
    \end{alertblock}
\end{frame}

\begin{frame}{Frame Title 2.5 (Algorithm 2)} % 示例 Frame 标题
    \begin{algorithm}[H]
        \caption{Placeholder Algorithm 2}
        \begin{algorithmic}[1]\scriptsize
            \For{each user $u \in U$}
            \State \colorbox{lightorange}{Limit records for user $u$ to $k$} \Comment{Placeholder comment}
            \EndFor
            \For{each silo $s \in S$}
                \State Perform local training
                \State Compute model update $\Delta_s^t$
                \State \colorbox{lightorange}{Use DP-SGD for local training} \Comment{Placeholder comment}
                \State $\Delta_s^t \leftarrow$ DP-SGD(...)
            \EndFor
            \State Aggregate updates
        \end{algorithmic}
    \end{algorithm}
    \vspace{-1em}
    \begin{alertblock}{Limitations:}
        \begin{columns}
            \column{0.5\textwidth}
            \begin{itemize}\scriptsize
                \item Limitation A
                \item Limitation B
            \end{itemize}

            \column{0.5\textwidth}
            \vspace{-2em}
            \begin{itemize}\scriptsize
                \item Limitation C
            \end{itemize}
            Summary of limitations.
        \end{columns}
        \end{alertblock}
\end{frame}

\subsection{Subsection Title 2.2} % 示例子章节标题


\begin{frame}{Frame Title 2.7 (Algorithm A vs B)} % 示例 Frame 标题
    \begin{columns}[T]
        \column{0.5\textwidth}
        \begin{algorithm}[H]
            \caption{Placeholder Algorithm A}
            \begin{algorithmic}[1]\tiny
                \For{each silo $s \in S$}
                    \For{each user $u \in U_s$}
                        \State Train using user $u$'s data
                        \For{\colorbox{lightorange}{each batch $B \in \mathcal{B}_u$}}
                            \State Compute batch gradient
                            \State Clip gradient
                            \State Weight gradient
                            \State Update local model
                        \EndFor
                        \State Compute final model update
                    \EndFor
                    \State Aggregate user updates
                    \State Add noise
                \EndFor
                \State Aggregate silo updates
            \end{algorithmic}
        \end{algorithm}

        \column{0.5\textwidth}
        \begin{algorithm}[H]
            \caption{Placeholder Algorithm B}
            \begin{algorithmic}[1]\tiny
                \For{each silo $s \in S$}
                    \For{each user $u \in U_s$}
                        \State Train using user $u$'s data
                        \State \colorbox{lightorange}{Compute full model update $\Delta_{s,u}^t$}
                        \State Clip update
                        \State Weight update
                    \EndFor
                    \State Aggregate user updates
                    \State Add noise
                \EndFor
                \State Aggregate silo updates
            \end{algorithmic}
        \end{algorithm}
    \end{columns}

    \colorbox{lightorange}{关键差异:}
    \vspace{-0.6em}
    \begin{itemize}\scriptsize
        \begin{columns}
            \column{0.5\textwidth}
            \item \textbf{梯度计算:}
                \begin{itemize}\tiny
                    \item AVG:Placeholder detail 1\cite{Author2020}
                    \item SGD:Placeholder detail 2
                \end{itemize}
            \column{0.5\textwidth}
            \item \textbf{内存使用:}
                \begin{itemize}\tiny
                    \item AVG:Placeholder detail 3\cite{origin}
                    \item SGD:Placeholder detail 4
                \end{itemize}
        \end{columns}
    \end{itemize}
\end{frame}

% ========== 结束页 (可以作为简单的致谢) ==========
\begin{frame}
    \centering
    \Huge 谢谢!!\cite{ConfAuthor2021}\\ % 这里的文本可以作为简单的致谢
    请各位批评指正!
\end{frame}

% ========== 参考文献页 ==========
% 注意:这里使用了 \bibliography{ref},需要一个名为 ref.bib 的文件
% 并且需要使用 bibtex 进行编译 (xelatex -> bibtex -> xelatex -> xelatex)
\begin{frame}[allowframebreaks]{参考文献}
    \bibliography{ref}
\end{frame}

\end{document}
